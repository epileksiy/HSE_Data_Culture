%!TEX TS-program = xelatex
\documentclass[12pt, a4paper, oneside]{article}

\usepackage{amsmath,amsfonts,amssymb,amsthm,mathtools}  % пакеты для математики

\usepackage[english, russian]{babel} % выбор языка для документа
\usepackage[utf8]{inputenc} % задание utf8 кодировки исходного tex файла
\usepackage[X2,T2A]{fontenc}        % кодировка

\usepackage{fontspec}         % пакет для подгрузки шрифтов
\setmainfont{Linux Libertine O}   % задаёт основной шрифт документа

\usepackage{unicode-math}     % пакет для установки математического шрифта
\setmathfont[math-style=upright]{Neo Euler} % шрифт для математики

% Конкретный символ из конкретного шрифта
% \setmathfont[range=\int]{Neo Euler}

% Конкретный символ из конкретного шрифта
% \setmathfont[range=\int]{Neo Euler}

%%%%%%%%%% Работа с картинками %%%%%%%%%
\usepackage{graphicx}                  % Для вставки рисунков
\usepackage{graphics}
\graphicspath{{images/}{pictures/}}    % можно указать папки с картинками
\usepackage{wrapfig}                   % Обтекание рисунков и таблиц текстом

%%%%%%%%%%%%%%%%%%%%%%%% Графики и рисование %%%%%%%%%%%%%%%%%%%%%%%%%%%%%%%%%
\usepackage{tikz, pgfplots}  % язык для рисования графики из latex'a

%%%%%%%%%% Гиперссылки %%%%%%%%%%
\usepackage{xcolor}              % разные цвета

\usepackage{hyperref}
\hypersetup{
	unicode=true,           % позволяет использовать юникодные символы
	colorlinks=true,       	% true - цветные ссылки, false - ссылки в рамках
	urlcolor=blue,          % цвет ссылки на url
	linkcolor=red,          % внутренние ссылки
	citecolor=green,        % на библиографию
	pdfnewwindow=true,      % при щелчке в pdf на ссылку откроется новый pdf
	breaklinks              % если ссылка не умещается в одну строку, разбивать ли ее на две части?
}


\usepackage{todonotes} % для вставки в документ заметок о том, что осталось сделать
% \todo{Здесь надо коэффициенты исправить}
% \missingfigure{Здесь будет Последний день Помпеи}
% \listoftodos --- печатает все поставленные \todo'шки

\usepackage[paper=a4paper, top=20mm, bottom=15mm,left=20mm,right=15mm]{geometry}
\usepackage{indentfirst}       % установка отступа в первом абзаце главы

\usepackage{setspace}
\setstretch{1.15}  % Межстрочный интервал
\setlength{\parskip}{4mm}   % Расстояние между абзацами
% Разные длины в латехе https://en.wikibooks.org/wiki/LaTeX/Lengths


\usepackage{xcolor} % Enabling mixing colors and color's call by 'svgnames'

\definecolor{MyColor1}{rgb}{0.2,0.4,0.6} %mix personal color
\newcommand{\textb}{\color{Black} \usefont{OT1}{lmss}{m}{n}}
\newcommand{\blue}{\color{MyColor1} \usefont{OT1}{lmss}{m}{n}}
\newcommand{\blueb}{\color{MyColor1} \usefont{OT1}{lmss}{b}{n}}
\newcommand{\red}{\color{LightCoral} \usefont{OT1}{lmss}{m}{n}}
\newcommand{\green}{\color{Turquoise} \usefont{OT1}{lmss}{m}{n}}

\usepackage{titlesec}
\usepackage{sectsty}
%%%%%%%%%%%%%%%%%%%%%%%%
%set section/subsections HEADINGS font and color
\sectionfont{\color{MyColor1}}  % sets colour of sections
\subsectionfont{\color{MyColor1}}  % sets colour of sections

%set section enumerator to arabic number (see footnotes markings alternatives)
\renewcommand\thesection{\arabic{section}.} %define sections numbering
\renewcommand\thesubsection{\thesection\arabic{subsection}} %subsec.num.

%define new section style
\newcommand{\mysection}{
	\titleformat{\section} [runin] {\usefont{OT1}{lmss}{b}{n}\color{MyColor1}} 
	{\thesection} {3pt} {} } 


%	CAPTIONS
\usepackage{caption}
\usepackage{subcaption}
%%%%%%%%%%%%%%%%%%%%%%%%
\captionsetup[figure]{labelfont={color=Turquoise}}

\pagestyle{empty}


%%%%%%%%%% Свои команды %%%%%%%%%%
\usepackage{etoolbox}    % логические операторы для своих макросов

% Все свои команды лучше всего определять не по ходу текста, как это сделано в этом документе, а в преамбуле!

% Одно из применений - уничтожение какого-то куска текста!
\newbool{answers}
%\booltrue{answers}
\boolfalse{answers}

\usepackage{enumitem}
% бульпоинты в списках
\definecolor{myblue}{rgb}{0, 0.45, 0.70}
\newcommand*{\MyPoint}{\tikz \draw [baseline, fill=myblue,draw=blue] circle (2.5pt);}
\renewcommand{\labelitemi}{\MyPoint}

% расстояние в списках
\setlist[itemize]{parsep=0.4em,itemsep=0em,topsep=0ex}
\setlist[enumerate]{parsep=0.4em,itemsep=0em,topsep=0ex}

\begin{document}

\section*{Самостоятельная работа номер 1. Вариант I}

\textbf{Решите все задания. Все ответы должны быть обоснованы. Все решения должны быть чётко приведены для каждого пункта. Рисунки должны быть чёткими, понятными. Все лнии должны быть подписаны. Списывание карается обнулением баллов. Удачи!}

\subsection*{Задача 1}

Серёжа любит мамбу. Он купил четыре конфеты, скушал и оценил вкус каждой по $10$-бальной шкале . Получилось $10$, $4$, $2$, $5$. Найдите средний вкус конфеты, найдите стандартное отклонение вкуса. Чему равна медиана? 

\subsection*{Задача 2}

Имеется пять чисел: $x$, $9$, $5$, $4$, $7$. При каком значении $x$ медиана будет равна среднему? 

\newpage 

\section*{Самостоятельная работа номер 1. Вариант II}

\textbf{Решите все задания. Все ответы должны быть обоснованы. Все решения должны быть чётко приведены для каждого пункта. Рисунки должны быть чёткими, понятными. Все лнии должны быть подписаны. Списывание карается обнулением баллов. Удачи!}

\subsection*{Задача 1}

Катя любит орешки. В лесу она собрала $5$ орехов и измерила вес каждого.  Получилось: $5, 9, 4, 1, 6$. Теперь она хочет посчитать по выборке пару описательных статистик. Помогите Маше найти среднее, медиану и стандартное отклонение. 

\subsection*{Задача 2}

Измерен рост $25$ человек. Средний рост оказался равным $160$ см. Медиана оказалась равной $155$ см. Машин рост в $163$ см был ошибочно внесен как $173$ см. Как изменится медиана и среднее после исправления ошибки? А как могут измениться медиана и среднее, если рост Маши равен $153$?

\newpage 

\section*{Самостоятельная работа номер 1. Вариант III}

\textbf{Решите все задания. Все ответы должны быть обоснованы. Все решения должны быть чётко приведены для каждого пункта. Рисунки должны быть чёткими, понятными. Все лнии должны быть подписаны. Списывание карается обнулением баллов. Удачи!}

\subsection*{Задача 1}

Дмитрий постоянно опаздывает. Причём часто на несколько дней. В один прекрасный момент он одумался и решил выписывать свои опоздания на бумажку. Последние опоздания составили Получилось $2, 2, 7, 6, 13$  дней.  Найдите среднее, медиану и моду опозданий Дмитрия.  Какие из них адекватно отражают типичное опоздание Дмитрия? Почему? 

\subsection*{Задача 2}

Имеется пять чисел: $x$, $9$, $5$, $4$, $7$. При каком значении $x$ медиана будет равна среднему? 

\newpage 

\section*{Самостоятельная работа номер 1. Вариант IV}

\textbf{Решите все задания. Все ответы должны быть обоснованы. Все решения должны быть чётко приведены для каждого пункта. Рисунки должны быть чёткими, понятными. Все лнии должны быть подписаны. Списывание карается обнулением баллов. Удачи!}

\subsection*{Задача 1}

Даня каждый день ходит на пары, чтобы считать ворон. В течение этой учебной недели он насчитал $5, 1, 4, 0, 5$ ворон. Сколько в среднем ворон видит Даня каждый день? Найдите моду и медиану числа ворон. Насколько адекватно каждая мера отражает типичное значение увиденных Даней ворон? Почему?

\subsection*{Задача 2}

Измерен рост $25$ человек. Средний рост оказался равным $160$ см. Медиана оказалась равной $155$ см. Машин рост в $163$ см был ошибочно внесен как $173$ см. Как изменится медиана и среднее после исправления ошибки? А как могут измениться медиана и среднее, если рост Маши равен $153$?

\newpage 


\section*{Самостоятельная работа номер 1. Вариант V}

\textbf{Решите все задания. Все ответы должны быть обоснованы. Все решения должны быть чётко приведены для каждого пункта. Рисунки должны быть чёткими, понятными. Все лнии должны быть подписаны. Списывание карается обнулением баллов. Удачи!}

\subsection*{Задача 1}

Ваня любит пить чай. Иногда он пьёт его с сахаром, иногда без. На этой неделе он помечал $1$ дни, когда пил чай с сахаром. Получилось $1,1,0,0,1,0$.  Найдите среднее значение сахарных дней в жизни Вани. Найдите стандартное отклонение сахарных дней. 

\subsection*{Задача 2}

Имеется пять чисел: $x$, $9$, $5$, $4$, $7$. При каком значении $x$ медиана будет равна среднему? 

\newpage 

\end{document} 

